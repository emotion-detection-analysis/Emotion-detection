%%%%%%%%%%%%%%%%%%%%%%%%%%%%%%%%%%%%%%%%%
% Beamer Presentation
% LaTeX Template
% Version 1.0 (10/11/12)
%
% This template has been downloaded from:
% http://www.LaTeXTemplates.com
%
% License:
% CC BY-NC-SA 3.0 (http://creativecommons.org/licenses/by-nc-sa/3.0/)
%
%%%%%%%%%%%%%%%%%%%%%%%%%%%%%%%%%%%%%%%%%

%----------------------------------------------------------------------------------------
%	PACKAGES AND THEMES
%----------------------------------------------------------------------------------------

\documentclass{beamer}

\mode<presentation> {

% The Beamer class comes with a number of default slide themes
% which change the colors and layouts of slides. Below this is a list
% of all the themes, uncomment each in turn to see what they look like.

%\usetheme{default}
%\usetheme{AnnArbor}
%\usetheme{Antibes}
%\usetheme{Bergen}
%\usetheme{Berkeley}
%\usetheme{Berlin}
%\usetheme{Boadilla}
%\usetheme{CambridgeUS}
%\usetheme{Copenhagen}
%\usetheme{Darmstadt}
%\usetheme{Dresden}
%\usetheme{Frankfurt}
%\usetheme{Goettingen}
%\usetheme{Hannover}
%\usetheme{Ilmenau}
%\usetheme{JuanLesPins}
%\usetheme{Luebeck}
\usetheme{Madrid}
%\usetheme{Malmoe}
%\usetheme{Marburg}
%\usetheme{Montpellier}
%\usetheme{PaloAlto}
%\usetheme{Pittsburgh}
%\usetheme{Rochester}
%\usetheme{Singapore}
%\usetheme{Szeged}
%\usetheme{Warsaw}

% As well as themes, the Beamer class has a number of color themes
% for any slide theme. Uncomment each of these in turn to see how it
% changes the colors of your current slide theme.

%\usecolortheme{albatross}
%\usecolortheme{beaver}
%\usecolortheme{beetle}
%\usecolortheme{crane}
%\usecolortheme{dolphin}
%\usecolortheme{dove}
%\usecolortheme{fly}
%\usecolortheme{lily}
%\usecolortheme{orchid}
%\usecolortheme{rose}
%\usecolortheme{seagull}
%\usecolortheme{seahorse}
%\usecolortheme{whale}
%\usecolortheme{wolverine}

%\setbeamertemplate{footline} % To remove the footer line in all slides uncomment this line
\setbeamertemplate{footline}[page number] % To replace the footer line in all slides with a simple slide count uncomment this line

\setbeamertemplate{navigation symbols}{} % To remove the navigation symbols from the bottom of all slides uncomment this line
}

\usepackage{graphicx} % Allows including images
\usepackage{booktabs} % Allows the use of \toprule, \midrule and \bottomrule in tables
%\usepackage {tikz}
\usepackage{tkz-graph}
\GraphInit[vstyle = Shade]
\graphicspath{{images/}}
\tikzset{
  LabelStyle/.style = { rectangle, rounded corners, draw,
                        minimum width = 2em, fill = yellow!50,
                        text = red, font = \bfseries },
  VertexStyle/.append style = { inner sep=5pt,
                                font = \normalsize\bfseries},
  EdgeStyle/.append style = {->, bend left} }
\usetikzlibrary {positioning}
%\usepackage {xcolor}
\definecolor {processblue}{cmyk}{0.96,0,0,0}
%----------------------------------------------------------------------------------------
%	TITLE PAGE
%----------------------------------------------------------------------------------------

\begin{document}
\title[Short title]{Emotion Detection Analysis} % The short title appears at the bottom of every slide, the full title is only on the title page

\author{Nirmal Suresh, Nanda Kishore, Sushmitha, Aravind} % Your name

\date{\today} % Date, can be changed to a custom date

\begin{frame}
\titlepage % Print the title page as the first slide
\end{frame}

\begin{frame}
\frametitle{Overview} % Table of contents slide, comment this block out to remove it
\tableofcontents % Throughout your presentation, if you choose to use \section{} and \subsection{} commands, these will automatically be printed on this slide as an overview of your presentation
\end{frame}

%----------------------------------------------------------------------------------------
%	PRESENTATION SLIDES
%----------------------------------------------------------------------------------------

%------------------------------------------------

\section{Interface Features}
\begin{frame}{Interface Features}
\begin{itemize}
\item Buttons for Train, Test, Evaluate User Input, and Exit
\item Buttons for uploading excel sheets (both train and test data)
\item Status bar to show current status of application (e.g. Evaluation Complete)
\item Bar chart to visualise emotion percentage
\end{itemize}
\end{frame}

\section{Training Interface}
\begin{frame}{Training Interface}
\begin{figure}[H]
\centering
\includegraphics[width=110mm]{TrainingGUI.png}
\end{figure}
\end{frame}

\section{Training Interface - Completed}
\begin{frame}{Training Interface - Completed}
\begin{figure}[H]
\centering
\includegraphics[width=110mm]{TrainingGUI-Complete.png}
\end{figure}
\end{frame}

\section{Testing Interface}
\begin{frame}{Testing Interface}
\begin{figure}[H]
\centering
\includegraphics[width=110mm]{TestingGUI.png}
\end{figure}
\end{frame}

\section{Testing Interface - Completed}
\begin{frame}{Testing Interface - Completed}
\begin{figure}[H]
\centering
\includegraphics[width=110mm]{TestingGUI-Complete.png}
\end{figure}
\end{frame}

\section{Data Visualisation 1}
\begin{frame}{Data Visualisation 1}
\begin{figure}[H]
\centering
\includegraphics[width=110mm]{DV1.png}
\end{figure}
\end{frame}

\section{Data Visualisation 2}
\begin{frame}{Data Visualisation 2}
\begin{figure}[H]
\centering
\includegraphics[width=110mm]{DV2.png}
\end{figure}
\end{frame}

\section{Data Visualisation 3}
\begin{frame}{Data Visualisation 3}
\begin{figure}[H]
\centering
\includegraphics[width=110mm]{DV3.png}
\end{figure}
\end{frame}

\section{Data Visualisation 4}
\begin{frame}{Data Visualisation 4}
\begin{figure}[H]
\centering
\includegraphics[width=110mm]{DV4.png}
\end{figure}
\end{frame}



\end{document}

%------------------------------------------------

\begin{frame}
\Huge{\centerline{Thank You}}
\end{frame}

%----------------------------------------------------------------------------------------
